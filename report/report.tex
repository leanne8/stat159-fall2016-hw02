\documentclass[]{article}
\usepackage{lmodern}
\usepackage{amssymb,amsmath}
\usepackage{ifxetex,ifluatex}
\usepackage{fixltx2e} % provides \textsubscript
\ifnum 0\ifxetex 1\fi\ifluatex 1\fi=0 % if pdftex
  \usepackage[T1]{fontenc}
  \usepackage[utf8]{inputenc}
\else % if luatex or xelatex
  \ifxetex
    \usepackage{mathspec}
    \usepackage{xltxtra,xunicode}
  \else
    \usepackage{fontspec}
  \fi
  \defaultfontfeatures{Mapping=tex-text,Scale=MatchLowercase}
  \newcommand{\euro}{€}
\fi
% use upquote if available, for straight quotes in verbatim environments
\IfFileExists{upquote.sty}{\usepackage{upquote}}{}
% use microtype if available
\IfFileExists{microtype.sty}{%
\usepackage{microtype}
\UseMicrotypeSet[protrusion]{basicmath} % disable protrusion for tt fonts
}{}
\ifxetex
  \usepackage[setpagesize=false, % page size defined by xetex
              unicode=false, % unicode breaks when used with xetex
              xetex]{hyperref}
\else
  \usepackage[unicode=true]{hyperref}
\fi
\usepackage[usenames,dvipsnames]{color}
\hypersetup{breaklinks=true,
            bookmarks=true,
            pdfauthor={Leanne Lee},
            pdftitle={Stat 159 HW 2},
            colorlinks=true,
            citecolor=blue,
            urlcolor=blue,
            linkcolor=magenta,
            pdfborder={0 0 0}}
\urlstyle{same}  % don't use monospace font for urls
\usepackage{graphicx,grffile}
\makeatletter
\def\maxwidth{\ifdim\Gin@nat@width>\linewidth\linewidth\else\Gin@nat@width\fi}
\def\maxheight{\ifdim\Gin@nat@height>\textheight\textheight\else\Gin@nat@height\fi}
\makeatother
% Scale images if necessary, so that they will not overflow the page
% margins by default, and it is still possible to overwrite the defaults
% using explicit options in \includegraphics[width, height, ...]{}
\setkeys{Gin}{width=\maxwidth,height=\maxheight,keepaspectratio}
\setlength{\parindent}{0pt}
\setlength{\parskip}{6pt plus 2pt minus 1pt}
\setlength{\emergencystretch}{3em}  % prevent overfull lines
\providecommand{\tightlist}{%
  \setlength{\itemsep}{0pt}\setlength{\parskip}{0pt}}
\setcounter{secnumdepth}{0}

\title{Stat 159 HW 2}
\author{Leanne Lee}
\date{October 5, 2016}

% Redefines (sub)paragraphs to behave more like sections
\ifx\paragraph\undefined\else
\let\oldparagraph\paragraph
\renewcommand{\paragraph}[1]{\oldparagraph{#1}\mbox{}}
\fi
\ifx\subparagraph\undefined\else
\let\oldsubparagraph\subparagraph
\renewcommand{\subparagraph}[1]{\oldsubparagraph{#1}\mbox{}}
\fi

\begin{document}
\maketitle

\subsection{Abstract}\label{abstract}

This homework is to reproduce the analysis from Section 3.1 of Ch 3
Linear Regression from the book ``An Introduction to Statistical
Learning'' (by James et al). The homework will analyze the advertising
dataset with linear regression and summary statistics.

\subsection{Introduction}\label{introduction}

From the dataset advertising, I can reproduce the simple linear
regression with TV and sales. The simple linear regression can predict a
quantitative response Y based on predictor variable X. With the
modeling, we can tell if there is a relationship exists between TV
advertising and Sales. If there exists a positive relationship, the
marketing team can decide to increase Tv advertising budget to promote
their sales.

\subsection{Data}\label{data}

The dataset \emph{Advertising.csv} comes from
\_``http://www-bcf.usc.edu/\textasciitilde{}gareth/ISL/Advertising.csv\_.
It consists for TV, Radio, Newspaper and Sales columns. The structure of
the columns are stored in numeric vectors.

\subsection{Methodology}\label{methodology}

By using the simple linear regression, we can predict the future sales
based on the amount of TV advertising.\\
The simple linear regression equation is the following:

\begin{verbatim}
Y = A + Bx + e
\end{verbatim}

Y = Sales A = Intercept Bx = TV advertising e = error

In R command, we can fit a linear regression model by using the
\emph{lm()} command. The null hypothesis is there does not exist a
relationship between TV ad and sales. The alternative hypothesis is that
there exists a relationship between TV ad and sales.

\subsection{Results}\label{results}

First, we inspect the historgram of sales and tv as the following:

\includegraphics{./images/histogram-sales.png}
\includegraphics{./images/histogram-tv.png}

Then we can export tables in Latex using xtable

\begin{verbatim}
load('./data/regression.RData')
library(xtable)
print(xtable(reg_summary, caption = "Regression Coefficients"), comment = FALSE)
\end{verbatim}

\includegraphics{./images/scatterplot-tv-sales.png} This scatter plot
shows that there is a positive correlation exists between TV ad and
sales.

\subsection{Conclusions}\label{conclusions}

This homework shows the relationship between TV advertising and sales.
There exists a positive correlation between the TV advertising and
sales. If the marketing team spend more money on TV advertising, they
will generate more sales. Based on the table, we can apply the same
linear regression model to radio and newspaper.

\end{document}
